%=================================================================================================
%= = = = = = = = = = = = = = = = = = = = = = = = = = = = = = = = = = = = = = = = = = = = = = = = =
%=																								 =
%=									 CONFIGURAÇÕES DO HYPERREF									 =
%=																								 =
%= = = = = = = = = = = = = = = = = = = = = = = = = = = = = = = = = = = = = = = = = = = = = = = = =
%=================================================================================================

% O hyperref é um pacote usado para construir remissões internas e hyper documento.
% \hypersetup O hyperref pode inserir informações dos dados do documento nos metadados do
% PDF final e também altera informações de cores dos links internos do documento final.

\definecolor{blue}{RGB}{41,5,195}
\makeatletter
\hypersetup{
     	%pagebackref=true,
		pdftitle={\@title}, 
		pdfauthor={\@author},
    	pdfsubject={\imprimirpreambulo},
	    pdfcreator={LaTeX with abnTeX2},
		pdfkeywords={abnt}{latex}{abntex}{abntex2}{trabalho acadêmico}, 
		colorlinks=true,       						%-> false: boxed links; true: colored links
    	linkcolor=blue,          					%-> color of internal links
    	citecolor=blue,        						%-> color of links to bibliography
    	filecolor=magenta,      					%-> color of file links
		urlcolor=blue,
		bookmarksdepth=4
}
\makeatother
\setlength{\parindent}{1.3cm}
\setlength{\parskip}{0.2cm}  
\makeindex


% ================================================================================================