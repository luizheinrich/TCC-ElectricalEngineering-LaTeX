%=================================================================================================
%=							       		      RESUMO							     		 	 =
%=================================================================================================


\begin{resumo}[Resumo]

Neste trabalho, é estudado um método de otimização e modelagem inversa de dispositivo nanofotônico para aplicação em sistemas de telecomunicações. Nas últimas décadas, o avanço tecnológico possibilitou um melhor entendimento da interação da luz com a matéria em escala nanométrica. Nesse contexto, surge a nanofotônica, uma área que vem atraindo muitas pesquisas, sobretudo, na fabricação de novos dispositivos que operam na faixa do terahertz. Se por um lado as simulações eletromagnéticas possibilitam a construção de novos dispositivos, por outro isso se torna um processo demorado à medida que aumenta a complexidade da resposta eletromagnética dessas nanoestruturas. Uma nova abordagem que vem surgindo é a modelagem inversa, isto é, modelar a geometria do dispositivo a partir da premissa de uma resposta em frequência com parâmetros ótimos de operação do dispositivo. Neste trabalho, é demonstrado o poderoso uso das redes neurais profundas no processo de modelagem inversa de dispositivos nanofotônicos. No total, quatro dispositivos foram submetidos ao método. Os resultados foram satisfatórios, em especial, quando a abordagem foi aplicada para dois circuladores de ressonância dipolo e quadrupolo baseados em cristal fotônico. Em ambos, a rede neural profunda teve êxito em realizar a modelagem inversa dos dispositivos, trabalho concluído em menos de um mês após a implementação. O método foi aplicado também a dois dispositivos baseados em grafeno, entretanto, o procedimento não obteve um resultado satisfatório. Neste último caso, a otimização ainda pode ser melhorada para os referidos dispositivos, por exemplo, investigando mais variáveis (da geometria ou do material) que podem não estar sendo consideradas no banco de dados da rede neural profunda. É importante como o aprendizado profundo está revolucionando muitas áreas da tecnologia e, no contexto da nanofotônica, também tem se mostrado uma ferramenta poderosa para o \textit{design} de nanoestruturas.


\vspace{\onelineskip}
\noindent
\textbf{Palavras-chave}: Nanofotônica. Inteligência Artificial. Otimização. Modelagem Inversa.
\end{resumo}