%=================================================================================================
%=							       		     ABSTRACT							     		 	 =
%=================================================================================================


\begin{resumo}[Abstract]
\begin{otherlanguage*}{english}

In this work, was developed optimization and inverse modeling method of nanophotonics devices for telecommunication system applications. In recent decades, technological advances have enabled a better understanding of the light-matter interaction at the nanometer scale. In this context, nanophotonics emerges as an area that has attracted a lot of research, especially in the design of new devices that operate in the terahertz range. As the electromagnetic simulations allow the study and design of these devices, on the other hand, that is a time-consuming process as their complexity increase. A new approach known as inverse modeling has been emerging in the last years, which consists in molding the device geometry from the optimal operating conditions in its frequency response. In this work, the powerful use of deep neural networks in the inverse modeling process of nanophotonic devices is demonstrated. Four devices were submitted by the method, and the results were particularly satisfactory when subjected to two dipole and quadrupole resonance circulators based on the photonic crystal. In both, the deep neural network was successful to predict the geometry of devices based solely on their target frequency response. The method to two graphene-based devices was also applied, and the results have shown that the procedure has not yet reached a satisfactory level. In the latter case, optimization can still be improved for those devices, for example, by investigating more variables (of geometry or material) that maybe are not being considered in the deep neural network database. It's significant how deep learning is revolutionizing many areas of technology and, in the context of nanophotonics, it has also proven to be a powerful tool for the design of nanostructures.


\vspace{\onelineskip}
\noindent
\textbf{Key words}: Nanophotonics. Artificial Intelligence. Optimization. Inverse Modeling.
\end{otherlanguage*}
\end{resumo}
