\chapter{Introdução}      \label{Introducao}

Muitas das descobertas na tecnologia foram possíveis a partir de um profundo entendimento das propriedades dos materiais \cite{Joannopoulos:08:Book}. Nesse sentido, os engenheiros aprenderam a fazer mais do que apenas manipular os materiais na sua forma bruta. Por exemplo, nas últimas décadas, uma nova fronteira foi alcançada a partir de vários estudos sobre o controle das propriedades óticas dos materiais. E é nesse contexto que surge a \textit{nanofotônica}, um ramo da engenharia ótica e engenharia elétrica que estuda o comportamento e interação da luz nos materiais em escala nanométrica. Compreender esses fenômenos óticos possibilita, sobretudo, a fabricação de novos dispositivos para atuarem como Circuitos Integrados (CI's) em sistemas de telecomunicações.

Nos últimos anos, houve um significativo crescimento de estudos de novos dispositivos que operam na faixa do \textit{Terahertz} (THz) do espectro eletromagnético (faixa de frequência de 0,1THz a 10THz), em especial, os dispositivos não-recíprocos tais como chaves, filtros, circuladores, divisores de potência e antenas \cite{dmitriev2012nonreciprocal,dmitriev2013optical,dmitriev2019dynamically}. A modelagem desses componentes não-recíprocos é feita através de simulações eletromagnéticas por meio do \textit{Método dos Elementos Finitos} (MEF) \cite{zienkiewicz1977finite}, ao passo que quanto mais complexos forem os componentes, maior é o tempo e o consumo de recursos computacionais \cite{noureen2021deep,valkanas2019neural}.

Por outro lado, a Inteligência Artificial (IA) tem revolucionado várias áreas da tecnologia, onde muitas aplicações emergiram nos últimos anos, como visão computacional \cite{krizhevsky2012imagenet}, carros autônomos \cite{al2017deep}, reconhecimento de fala \cite{hinton2012deep}, processamento de linguagem natural \cite{socher2013recursive}, reconhecimento facial \cite{taigman2014deepface}, etc. Nesse sentido, a aplicação de Redes Neurais Artificiais (RNA) para a modelagem de dispositivos fotônicos cresceu significativamente nos últimos anos, tendo em vista que se aplicam muito bem aos problemas multivariáveis e não-lineares \cite{noureen2021deep,malkiel2017deep,liu2018training,Kojima2020sep}.

Usar a inteligência artificial para modelar um dispositivo a partir de uma condição ideal (condições desejáveis de operação do dispositivo) é uma abordagem conhecida como \textit{modelagem inversa} (ou ainda, \textit{design inverso}) \cite{liu2018training,peurifoy2018nanophotonic,Kojima2020sep}. Neste caso, a modelagem direta é quando se executa o projeto de forma convencional, isto é, quando se constrói a geometria do dispositivo nanofotônico para então se obter a sua resposta em frequência (onde serão avaliados os parâmetros de operação e desempenho do dispositivo). Esse processo deve ser repetido tantas vezes quanto forem necessárias até chegar-se numa resposta em frequência considerada ótima. Na abordagem \textit{modelagem inversa}, a geometria do dispositivo não é definida por primeiro. Ao invés disso, constrói-se uma resposta em frequência com condições ideais de operação do dispositivo. Desta forma, é montado um banco de dados com vários exemplos aleatórios de parâmetros de geometria associados com a respectiva resposta em frequência. Após aprender, a partir do banco de dados, os princípios entre os parâmetros geométricos e a resposta em frequência, a rede neural poderá obter a geometria apropriada que está relacionada com a resposta em frequência desejada.

O presente trabalho se propôs à aplicação de Redes Neurais Profundas (do inglês: \textit{Deep Neural Networks (DNNs)}) para a modelagem inversa de dispositivos nanofotônicos. Nessa abordagem, é mostrado como as DNNs podem agilizar o processo de \textit{design} e fornecer uma capacidade de caracterização robusta e eficiente de nanoestruturas complexas com base em uma resposta em frequência desejada, em um tempo significativamente menor que os métodos convencionais. Os dispositivos submetidos a esse processo são dois circuladores baseados em cristais fotônicos (discutidos em \cite{DMITRIEV2021100954}) e dois divisores de potência baseados em grafeno (discutidos em \cite{Dmitriev2021Nonreciprocal} (ver Apêndice \ref{Apendice})). Todos os dispositivos são não-recíprocos e operam na região do terahertz e subterahertz.


\section{Trabalhos Relacionados}      \label{Trabalhos Relacionados}

Muitos trabalhos no âmbito da modelagem inversa de geometria de dispositivos fotônicos foram desenvolvidos nos últimos anos. A proposta, em comum nesses estudos, é muito parecida com a que é abordada neste presente trabalho: utilizar redes neurais profundas para modelar a geometria de dispositivos fotônicos e nanofotônicos a partir de uma resposta em frequência com características desejadas de operação.

No estudo discutido em \cite{liu2018training}, os autores desenvolveram uma arquitetura de rede neural para a modelagem inversa de dispositivos nanofotônicos, ao mesmo tempo, em que visa solucionar o problema de não-unicidade da resposta eletromagnética. Isto é, no problema abordado pelos autores, a resposta eletromagnética do dispositivo estudado não é única, pois várias configurações de geometria de dispositivo podem resultar na mesma resposta eletromagnética. Essa categoria de problema torna muito difícil o treinamento de redes neurais a partir de um banco de dados muito extenso, pois gera conflitos nas instâncias de treinamento, como o utilizado pelos autores, contendo 500 mil instâncias. Como solução, os autores propuseram uma arquitetura de rede com duas estruturas características: \textit{rede direta} e a \textit{rede inversa}, ambas com uma única camada intermediária. A \textit{rede direta} é responsável por mapear as relações da geometria com o espectro. Posteriormente, a rede \textit{rede inversa} é alimentada com o espectro (concatenada com a saída da \textit{rede direta}) e produz em sua saída as geometrias associadas. Desta forma, é demonstrado que a arquitetura de rede proposta tolera instâncias de treinamento não exclusivas explícitas e implícitas, além de fornecer uma maneira de treinar grandes redes neurais para o projeto de modelagem inversa de estruturas fotônicas complexas.

Em \cite{malkiel2017deep}, os autores propuseram a resolução do \textit{design inverso} de nanoestruturas em metasuperfícies definidas por alguns parâmetros através de uma rede neural bidirecional. Nesse sentido, foram desenvolvidas duas redes neurais profundas, sendo uma rede de previsão de geometria (chamada de \textit{GPN}) e uma rede de previsão de espectro (chamada de \textit{SPN}). A rede GPN prevê os parâmetros geométricos dado uma resposta espectral, enquanto a SPN mapeia esse espectro para com a geometria de entrada da estrutura estudada. Durante o processo de treinamento, a saída da GPN é alimentada na SPN (processo parecido com o estudo em \cite{liu2018training}). Dado um par de treinamento composto por parâmetros geométricos e seu espectro correspondente, o objetivo é minimizar a perda entre o par de treinamento e as saídas da GPN e SPN. Após o treinamento da rede bidirecional, novos projetos de modelagem inversa com várias respostas espectrais desejadas podem ser gerados rapidamente alimentando os objetivos na GPN. Outro mecanismo que os autores implementaram foi uma rede neural que compõe uma \textit{estrutura de pré-processamento}. Na avaliação dos autores, a inclusão do pré-processamento demonstrou um melhor desempenho quando avaliado com diferentes arquiteturas de rede.

Em \cite{tahersima2019deep}, os autores implementaram Redes Neurais profundas (do inglês: \textit{Deep Neural Network} (DNN)) para a modelagem inversa de um dispositivo divisor de potência por dois (1x2) baseado em cristal fotônico. Todo o processo foi implementado com o uso do \textit{framework} TensorFlow, no ambiente da linguagem de programação Python. No referido estudo, os autores apresentam um exemplo de nanocavidade em uma heteroestrutura de cristal fotônico 2D. O objetivo da otimização é identificar as posições dessas cavidades de modo a maximizar os fatores de qualidade dado uma certa estrutura inicial. O banco de dados usado para alimentar a DNN foi na ordem de 20.000 instâncias e o \textit{espaço de design} na ordem de $2^{400}$. Nesse sentido, os autores demonstram que as DNNs, contando com sua capacidade de processar dados volumosos e de grande dimensão do \textit{espaço de design} (conceito explicado à diante), tornaram-se uma arquitetura indispensável para o projeto de dispositivos fotônicos com alta complexidade geométrica.

Outra abordagem de modelagem inversa é discutida em \cite{liu2021tackling}, onde os autores introduzem os rápidos avanços nas técnicas de \textit{machine learning} e suas aplicações no processo de modelagem e otimização das estruturas fotônicas. Assim, são avaliados diferentes \textit{graus de liberdade} (do inglês: \textit{Degree of Freedom} (DOF)), fator atribuído conforme a \textit{liberdade} que as técnicas de modelagem permitem ao projetista. Todas as possibilidades de configurações e arranjos de geometria é definido como um \textit{espaço de design} e o número de variáveis consideradas corresponde à dimensionalidade desse espaço.

Por exemplo, para modelagens com \textit{soluções analíticas} ou \textit{varredura paramétrica simples}, implica em um baixo DOF, pois as combinações dos parâmetros de \textit{design} não permitem explorar um espaço grande de soluções. No sentido crescente do DOF, encontaram-se as soluções por meio do \textit{design inverso} (escopo deste trabalho), onde o uso de técnicas em aprendizado de máquina permite explorar um espaço maior de soluções (isto é, crescimento da dimensionalidade do \textit{espaço de design}). Nesse contexto, são usados modelo discriminativo para capturar as relações entre parâmetros de \textit{design} e respostas ópticas com quantidades reduzida de dados. Deve-se notar que, dado que múltiplas configurações de estruturas podem corresponder à mesma resposta ótica (problema da \textit{não-unicidade}), de forma que um único modelo discriminativo não é capaz de mapear perfeitamente uma resposta ótica de volta a um conjunto único de parâmetros de \textit{design}. São necessárias estratégias adicionais de treinamento se modelos discriminatórios forem usados para a otimização e \textit{design}.

Quando o DOF continua crescendo, modelos generativos podem ajudar a reduzir a dimensionalidade do \textit{espaço de design}, de forma a buscar relações entre parâmetros de \textit{design} e respostas óticas para maior otimização. Os modelos generativos podem ser aproveitados conjuntamente com modelos discriminativos, bem como algoritmos tradicionais de otimização para acelerar o processo de \textit{design} ou localizar as soluções ideais globais. Uma abordagem desse alto DOF são as que utilizam o aprendizado não-supervisionado, fazendo o uso de Redes Neurais Adversárias Generativas (do inglês: \textit{Generative Adversarial Network (GAN)}) \cite{SoRho201912551261}.


\section{Organização do Trabalho}

Este trabalho está organizado como se segue.

O Capítulo \ref{Revisao Bibliografica} é dividido em duas principais partes: na Seção \ref{Nanofotonica} é feita uma revisão dos conceitos básicos da nanofotônica; e na Seção \ref{Machine Learning} são apresentados os conceitos básicos de redes neurais artificiais.

No Capítulo \ref{Metodo Proposto} é vista a metodologia de aplicação do procedimento de modelagem inversa e otimização.

O Capítulo \ref{Resultados} mostra os resultados obtidos após a implementação do método de otimização proposto neste trabalho.

Por fim, no Capítulo \ref{Consideracoes Finais}, são mostradas as considerações finais desde trabalho, onde perpassa pela \textit{discussão}, \textit{conclusão} e \textit{sugestão para trabalhos futuros}.

Em adicional, o Apêndice \ref{Apendice} é apresentado em duas partes: a Seção \ref{Aplicacao Divisor} mostra todo o estudo do presente trabalho aplicado a outros dois dispositivos divisores de potência baseados em grafeno; e a Seção \ref{Aplicacao Divisor} mostra o detalhamento dos parâmetros de geometria dos dispositivos baseados em cristal fotônico, os hiperparâmetros da Rede Neural e as configurações do computador no qual o ambiente de otimização foi executado.