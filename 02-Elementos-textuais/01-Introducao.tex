\chapter{Introdução}      \label{Introducao}

A \textit{nanofotônica} é um ramo da engenharia ótica que estuda o comportamento da luz em escala nanométrica. Compreender esses fenômenos óticos permite construir dispositivos a partir destes nesse contexto, há uma grande linha de pesquisa em dispositivos fotônicos para atuarem em circuitos puramente óticos.


\section{Estado da Arte}

Nesse contexto da otimização de dispositivos nanofotônicos, alguns estudos emergiram nos últimos anos com essa proposta.


\section{Motivação}

A inteligência artificial tem revolucionado muitos campos de estudo. Na engenharia,.


\section{Objetivos}

At vero eos et accusamus et iusto odio dignissimos ducimus qui blanditiis praesentium voluptatum deleniti atque corrupti quos dolores et quas molestias excepturi sint occaecati cupiditate non provident, similique sunt in culpa qui officia deserunt mollitia animi, id est laborum et dolorum fuga. Et harum quidem rerum facilis est et expedita distinctio. Nam libero tempore, cum soluta nobis est eligendi optio cumque nihil impedit quo minus id quod maxime placeat facere possimus, omnis voluptas assumenda est, omnis dolor repellendus. Temporibus autem quibusdam et aut officiis debitis aut rerum necessitatibus saepe eveniet ut et voluptates repudiandae sint et molestiae non recusandae. Itaque earum rerum hic tenetur a sapiente delectus, ut aut reiciendis voluptatibus maiores alias consequatur aut perferendis doloribus asperiores repellat.


\section{Materiais e Métodos}

Para o estudo proposto neste trabalho, .


\section{Organização do Trabalho}

Este trabalho está organizado como se segue. O Capítulo. Por fim, no Capítulo \ref{Consideracoes Finais}, as considerações finais desde trabalho. O leitor também poderá consultar os manuscritos dos cálculos de projeto no Apêndice.