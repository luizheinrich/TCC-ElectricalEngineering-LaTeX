\chapter{Introdução}      \label{Introducao}

A \textit{nanofotônica} é um ramo da engenharia ótica que estuda o comportamento da luz em escala nanométrica. Compreender esses fenômenos óticos permite construir dispositivos a partir destes nesse contexto, há uma grande linha de pesquisa em dispositivos fotônicos para atuarem em circuitos puramente óticos \cite{luizheinrich2021}.


\section{Estado da Arte}

Nesse contexto da otimização de dispositivos nanofotônicos, alguns estudos emergiram nos últimos anos com essa proposta.


\section{Motivação}

A inteligência artificial tem revolucionado muitos campos de estudo. Na engenharia,.


\section{Objetivos}

Um dos escopos deste trabalho de conclusão de curso é servir de referência bibliográfica e suporte aos estudos futuros sobre o processo de otimização de dispositivos por meio de \textit{machine learning}.

Como objetivo geral, este trabalho visa.

Como objetivo específico, este trabalho.


\section{Materiais e Métodos}

Compõe a metodologia deste trabalho a revisão bibliográfica e os métodos experimentais usados para viabilizar e validar o estudo. Foram usados softwares de simulação numérica e de cálculo numérico. 


\section{Organização do Trabalho}

Este trabalho está organizado como se segue. O Capítulo. Por fim, no Capítulo \ref{Consideracoes Finais}, as considerações finais desde trabalho. O leitor também poderá consultar os manuscritos dos cálculos de projeto no Apêndice.