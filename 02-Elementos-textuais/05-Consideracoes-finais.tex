\chapter{Considerações Finais}

\section{Conclusão}

Neste trabalho de pesquisa, foi proposto um procedimento de otimização de dispositivos nanofotônicos com o uso algoritmos em \textit{Machine Learning}.



\section{Trabalhos Atuais e Sugestões Para Trabalhos Futuros}

Dapibus gravida tristique sodales purus condimentum porttitor, aliquam vulputate condimentum donec sapien justo praesent, sociosqu pellentesque dictum eros auctor. odio amet sem pretium eros facilisis curabitur velit tempus sapien, sodales praesent rutrum interdum tincidunt habitant euismod augue, tristique vehicula tempus molestie at quisque erat potenti. lacinia pulvinar class dictumst suspendisse eget etiam, molestie lectus class aenean purus eros primis, quam purus lectus viverra est. ante eget pretium lacus torquent cras ullamcorper neque, elit platea diam nulla potenti class auctor lectus, tempor dapibus a justo aptent rhoncus. praesent aliquet purus felis nostra pellentesque odio quisque praesent porttitor, curae maecenas placerat nostra maecenas erat ac tristique, iaculis porttitor habitant aptent suscipit posuere accumsan curabitur. 

Himenaeos rutrum augue nec nunc vulputate senectus vel aptent blandit, curae pulvinar gravida enim condimentum pretium ante posuere vehicula, pellentesque ut dolor amet ante cras cubilia neque. laoreet aliquet rutrum eros mattis torquent curae habitasse, pulvinar turpis nulla convallis molestie netus tincidunt, habitant et ut integer inceptos massa. sapien etiam sed posuere viverra ullamcorper rutrum euismod, platea netus imperdiet ultrices feugiat lectus sit, cursus rutrum tincidunt mollis risus ligula. dui quisque sapien tellus curabitur proin lacus proin lorem, magna aliquam adipiscing dictum leo consequat nisl orci, etiam vitae mi eros augue mauris imperdiet.